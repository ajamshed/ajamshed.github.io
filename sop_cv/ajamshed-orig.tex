%%%%%%%%%%%%%%%%%%%%%%%%%%%%%%%%%%%%%%%%%%%%%%%%%%%%%%%%%%%%%%%%%%%%%%%%
%%%%%%%%%%%%%%%%%%%%%% Simple LaTeX CV Template %%%%%%%%%%%%%%%%%%%%%%%%
%%%%%%%%%%%%%%%%%%%%%%%%%%%%%%%%%%%%%%%%%%%%%%%%%%%%%%%%%%%%%%%%%%%%%%%%

%%%%%%%%%%%%%%%%%%%%%%%%%%%%%%%%%%%%%%%%%%%%%%%%%%%%%%%%%%%%%%%%%%%%%%%%
%% NOTE: If you find that it says                                     %%
%%                                                                    %%
%%                           1 of ??                                  %%
%%                                                                    %%
%% at the bottom of your first page, this means that the AUX file     %%
%% was not available when you ran LaTeX on this source. Simply RERUN  %% 
%% LaTeX to get the ``??'' replaced with the number of the last page  %% 
%% of the document. The AUX file will be generated on the first run   %%
%% of LaTeX and used on the second run to fill in all of the          %%
%% references.                                                        %%
%%%%%%%%%%%%%%%%%%%%%%%%%%%%%%%%%%%%%%%%%%%%%%%%%%%%%%%%%%%%%%%%%%%%%%%%

%%%%%%%%%%%%%%%%%%%%%%%%%%%% Document Setup %%%%%%%%%%%%%%%%%%%%%%%%%%%%

% Don't like 10pt? Try 11pt or 12pt
\documentclass[10pt]{article}

% This is a helpful package that puts math inside length specifications
\usepackage{calc}

% Layout: Puts the section titles on left side of page
\reversemarginpar

%
%         PAPER SIZE, PAGE NUMBER, AND DOCUMENT LAYOUT NOTES:
%
% The next \usepackage line changes the layout for CV style section
% headings as marginal notes. It also sets up the paper size as either
% letter or A4. By default, letter was used. If A4 paper is desired,
% comment out the letterpaper lines and uncomment the a4paper lines.
%
% As you can see, the margin widths and section title widths can be
% easily adjusted.
%
% ALSO: Notice that the includefoot option can be commented OUT in order
% to put the PAGE NUMBER *IN* the bottom margin. This will make the
% effective text area larger.
%
% IF YOU WISH TO REMOVE THE ``of LASTPAGE'' next to each page number,
% see the note about the +LP and -LP lines below. Comment out the +LP
% and uncomment the -LP.
%
% IF YOU WISH TO REMOVE PAGE NUMBERS, be sure that the includefoot line
% is uncommented and ALSO uncomment the \pagestyle{empty} a few lines
% below.
%

%% Use these lines for letter-sized paper
\usepackage[paper=letterpaper,
            %includefoot, % Uncomment to put page number above margin
            marginparwidth=1.2in,     % Length of section titles
            marginparsep=.05in,       % Space between titles and text
            margin=1in,               % 1 inch margins
            includemp]{geometry}

%% Use these lines for A4-sized paper
%\usepackage[paper=a4paper,
%            %includefoot, % Uncomment to put page number above margin
%            marginparwidth=30.5mm,    % Length of section titles
%            marginparsep=1.5mm,       % Space between titles and text
%            margin=25mm,              % 25mm margins
%            includemp]{geometry}

%% More layout: Get rid of indenting throughout entire document
\setlength{\parindent}{0in}

%% This gives us fun enumeration environments. compactitem will be nice.
\usepackage{paralist}

%% Reference the last page in the page number
%
% NOTE: comment the +LP line and uncomment the -LP line to have page
%       numbers without the ``of ##'' last page reference)
%
% NOTE: uncomment the \pagestyle{empty} line to get rid of all page
%       numbers (make sure includefoot is commented out above)
%
\usepackage{fancyhdr,lastpage}
\pagestyle{fancy}
%\pagestyle{empty}      % Uncomment this to get rid of page numbers
\fancyhf{}\renewcommand{\headrulewidth}{0pt}
\fancyfootoffset{\marginparsep+\marginparwidth}
\newlength{\footpageshift}
\setlength{\footpageshift}
          {0.5\textwidth+0.5\marginparsep+0.5\marginparwidth-2in}
\lfoot{\hspace{\footpageshift}%
       \parbox{4in}{\, \hfill %
                    \arabic{page} of \protect\pageref*{LastPage} % +LP
%                    \arabic{page}                               % -LP
                    \hfill \,}}

% Finally, give us PDF bookmarks
\usepackage{color,hyperref}
\definecolor{darkblue}{rgb}{0.0,0.0,0.3}
\hypersetup{colorlinks,breaklinks,
            linkcolor=darkblue,urlcolor=darkblue,
            anchorcolor=darkblue,citecolor=darkblue}

%%%%%%%%%%%%%%%%%%%%%%%% End Document Setup %%%%%%%%%%%%%%%%%%%%%%%%%%%%


%%%%%%%%%%%%%%%%%%%%%%%%%%% Helper Commands %%%%%%%%%%%%%%%%%%%%%%%%%%%%

% The title (name) with a horizontal rule under it
%
% Usage: \makeheading{name}
%
% Place at top of document. It should be the first thing.
\newcommand{\makeheading}[1]%
        {\hspace*{-\marginparsep minus \marginparwidth}%
         \begin{minipage}[t]{\textwidth+\marginparwidth+\marginparsep}%
                {\large \bfseries #1}\\[-0.15\baselineskip]%
                 \rule{\columnwidth}{1pt}%
         \end{minipage}}

% The section headings
%
% Usage: \section{section name}
%
% Follow this section IMMEDIATELY with the first line of the section
% text. Do not put whitespace in between. That is, do this:
%
%       \section{My Information}
%       Here is my information.
%
% and NOT this:
%
%       \section{My Information}
%
%       Here is my information.
%
% Otherwise the top of the section header will not line up with the top
% of the section. Of course, using a single comment character (%) on
% empty lines allows for the function of the first example with the
% readability of the second example.
\renewcommand{\section}[2]%
        {\pagebreak[2]\vspace{1.3\baselineskip}%
         \phantomsection\addcontentsline{toc}{section}{#1}%
         \hspace{0in}%
         \marginpar{
         \raggedright \scshape #1}#2}

% An itemize-style list with lots of space between items
\newenvironment{outerlist}[1][\enskip\textbullet]%
        {\begin{itemize}[#1]}{\end{itemize}%
         \vspace{-.6\baselineskip}}

% An environment IDENTICAL to outerlist that has better pre-list spacing
% when used as the first thing in a \section 
\newenvironment{lonelist}[1][\enskip\textbullet]%
        {\vspace{-\baselineskip}\begin{list}{#1}{%
        \setlength{\partopsep}{0pt}%
        \setlength{\topsep}{0pt}}}
        {\end{list}\vspace{-.6\baselineskip}}

% An itemize-style list with little space between items
\newenvironment{innerlist}[1][\enskip\textbullet]%
        {\begin{compactitem}[#1]}{\end{compactitem}}

% To add some paragraph space between lines.
% This also tells LaTeX to preferably break a page on one of these gaps
% if there is a needed pagebreak nearby.
\newcommand{\blankline}{\quad\pagebreak[2]}

%%%%%%%%%%%%%%%%%%%%%%%% End Helper Commands %%%%%%%%%%%%%%%%%%%%%%%%%%%

%%%%%%%%%%%%%%%%%%%%%%%%% Begin CV Document %%%%%%%%%%%%%%%%%%%%%%%%%%%%

\begin{document}
\makeheading{M{\footnotesize UHAMMAD} A{\footnotesize SIM} J{\footnotesize AMSHED}}

\section{Contact Information}
%
% NOTE: Mind where the & separators and \\ breaks are in the following
%       table.
%
% ALSO: \rcollength is the width of the right column of the table 
%       (adjust it to your liking; default is 1.85in).
%
\newlength{\rcollength}\setlength{\rcollength}{2.40in}%
%
\begin{tabular}[t]{@{}p{\textwidth-\rcollength}p{\rcollength}}
3245 E3 Building 
                          %& \textit{Voice:} (412) 296-2532 \\
EE Dept.        & \textit{E-mail:}
\href{mailto:ajamshed@ndsl.kaist.edu}{ajamshed@ndsl.kaist.edu}\\
KAIST       & \textit{WWW:}
\href{http://www.ndsl.kaist.edu/~ajamshed}{www.ndsl.kaist.edu/$\sim$ajamshed}\\
335 Gwahangno, Yuseong-gu, Daejeon 205-701 \\
Republic of Korea    
\end{tabular}

%\section{Security Clearance} 
%
%Department of Defense Top Secret SCI with polygraph (expired: 2002) 

%\section{Citizenship}
%
%USA

\section{Research Interests}
%
Networked Computer Systems, Computer $\&$ Network Security, and Distributed Storage Systems 

\section{Education}
%
\href{http://www.kaist.ac.kr}{\textbf{Korea Advanced Institute of Science \& Technology (KAIST)}}\\
Daejeon, Republic of Korea
\begin{outerlist}

\item[] Graduate Student, 
        \href{http://www.ee.kaist.ac.kr/}
             {Electrical Engineering} 
        (Sept '10-onwards)
        \begin{innerlist}
        \item Advisor: 
              \href{http://www.ndsl.kaist.edu/~kyoungsoo/}
                   {Dr. KyoungSoo Park} \\
        \end{innerlist}
\end{outerlist}

\href{http://www.pitt.edu/}{\textbf{University of Pittsburgh}},
Pittsburgh, Pennsylvania, USA
\begin{outerlist}

\item[] MS, 
        \href{http://www.cs.pitt.edu/}
             {Computer Science} 
        (Apr 2010)
        \begin{innerlist}
        \item Advisor: 
              \href{http://www.ndsl.kaist.edu/~kyoungsoo/}
                   {Dr. KyoungSoo Park} \\
        \end{innerlist}
\end{outerlist}

\href{http://www.lums.edu.pk/}{\textbf{Lahore University of Management Sciences}},
Lahore, Pakistan
\begin{outerlist}
\item[] BSc (Hons), 
        \href{http://cs.lums.edu.pk/}
             {Computer Science}, (May 2005)
        \begin{innerlist}
        \item Minor in Mathematics 
        \end{innerlist}
\end{outerlist}

%\section{Awards} 
%
%\href{http://www.nsf.gov/}{National Science Foundation}
%\begin{innerlist}
%\item \href{http://www.nsfgk12.org/}{GK-12 Fellowship}, 2006
%\item \href{http://www.nsf.gov/grfp}
%           {Graduate Research Fellowship} Honorable Mention, 2005
%\end{innerlist}

%\blankline

%\href{http://www.osu.edu}{The Ohio State University}
%\begin{innerlist}
%\item \href{http://www.gradsch.osu.edu/Content.aspx?Content=44&itemid=2}
%           {Dean's Distinguished University Fellowship}, 2004
%\item Electrical and Computer Engineering Bradshaw Scholarship,
%        2002--2004
%\item Electrical and Computer Engineering Shafstall Scholarship,
%        2001--2003
%\item University Scholarship, 1999--2003
%\end{innerlist}

\section{Research Experience}
\textbf{Networked \& Distributed Systems Lab}, {\small EE Dept.,
KAIST}% 
				\hfill \textbf{Fall '10-onwards} \\
\textit{Graduate Researcher \\
(i) Human Detection in the Internet: See [1] and the Projects section for more details. \\
(ii) High Speed Session Layer I/O \\
(iii) Large Scale Intrusion Detection Systems (IDS)} \\

\textbf{Distributed Systems Lab}, {\small CS Dept.,
Univ of Pittsburgh}% 
				\hfill \textbf{Summer '09-Spring '10} \\
\textit{Graduate Researcher \\
(i) Email Spam Behavior - Detection \& Prevention: Analyzed email spamming behaviors as 
seen by honeypots in open-proxy settings.  \\
(ii) Human Detection in the Internet: See [1] and the Projects section for more details.} \\

\textbf{Network Systems Lab}, {\small CS Dept.,
Univ of Pittsburgh}% 
				\hfill \textbf{Summers '07 $\&$ '08} \\
\textit{Graduate Researcher - Worked on a framework to mitigate the effects 
			of application-level DDoS attacks [2,3]. See Projects section for 
more details.} \\

\section{Publications}
[1] Jamshed, M., Go, Y., Park, K.
{ ``Supressing Malicious Bot Traffic using an Accurate Human Attester." }
     $8^{th}$ USENIX Symposium on Networked Systems Design and Implementation (NSDI 2011) 
     (Poster)\\
     
[2] Jamshed, M., Kim, W., Park, K. 
\href{http://www.ndsl.kaist.edu/~ajamshed/papers/apsys10.pdf}
     { ``Suppressing Bot Traffic with Accurate Human Attestations." }
     $1^{st}$ ACM Asia-Pacific Workshop on Systems (ApSys 2010) 
     held in conjunction with SIGCOMM 2010\\

[3] Djalaliev, P., Jamshed, M., Farnan, N., Brustoloni, J.C. 
	  \href{http://www.ndsl.kaist.edu/~ajamshed/papers/icccn08.pdf}
	    {``Sentinel: Hardware-Accelerated Mitigation of Bot-Based DDoS Attacks.''}
	  IEEE IC3N 2008 Network Security Track \\

[4] Jamshed, M., Brustoloni, J. ``In-Network Server-Directed Client Authentication 
		and Packet Classification." $35^{th}$ Annual IEEE Conference on Local
                Computer Networks (LCN) 2010\\

\section{Teaching Experience}
\href{http://www.kaist.ac.kr}{\textbf{Korea Advanced Institute of Science \& Technology (KAIST)}} \\
\textit{Teaching Assistant}, EE Dept. \\
\\
Led weekly precepts and graded assignments for the 
following courses:

\begin{innerlist}
\item \href{http://www.ndsl.kaist.edu/~kyoungsoo/ee209/index.shtml}
           {EE 209: Programming Structures for Electrical Engineering}%
			\hfill {Falls \{2010 \& 2011\}} \\
%\item \href{http://www.cs.pitt.edu/undergrad/courses/cs0007.php}
%           {CS 0007: Introduction to Computer Programming}%
%			\hfill {Falls \{2008, 2007 \& 2006\}} \\
%\item \href{http://www.cs.pitt.edu/undergrad/courses/cs0449.php}
%           {CS 0449: Introduction to Systems Software}%
%			\hfill {Spring 2008}
%\item \href{http://www.cs.pitt.edu/undergrad/courses/cs0007.php}
%           {CS 0007: Introduction to Computer Programming}%
%			\hfill {Fall 2007}
%\item \href{http://www.cs.pitt.edu/undergrad/courses/cs0007.php}
%           {CS 0007: Introduction to Computer Programming}%
%			\hfill {Fall 2006} \\
\end{innerlist}

\href{http://www.pitt.edu}{\textbf{University of Pittsburgh}} \\
\textit{Teaching Assistant}, CS Dept. \\
\\
My main responsibilities have ranged from leading weekly recitations
and grading assignments to making labs for the following courses:

\begin{innerlist}
\item \href{http://www.cs.pitt.edu/undergrad/courses/cs0449.php}
           {CS 0449: Introduction to Systems Software}%
			\hfill {Springs \{2009 \& 2008\}}
\item \href{http://www.cs.pitt.edu/undergrad/courses/cs0007.php}
           {CS 0007: Introduction to Computer Programming}%
			\hfill {Falls \{2008, 2007 \& 2006\}} \\
%\item \href{http://www.cs.pitt.edu/undergrad/courses/cs0449.php}
%           {CS 0449: Introduction to Systems Software}%
%			\hfill {Spring 2008}
%\item \href{http://www.cs.pitt.edu/undergrad/courses/cs0007.php}
%           {CS 0007: Introduction to Computer Programming}%
%			\hfill {Fall 2007}
%\item \href{http://www.cs.pitt.edu/undergrad/courses/cs0007.php}
%           {CS 0007: Introduction to Computer Programming}%
%			\hfill {Fall 2006} \\
\end{innerlist}

\textit{Course Grader}, CS Dept.
\begin{innerlist}
\item \href{http://www.cs.pitt.edu/undergrad/courses/cs1550.php}
           {CS 1550: Introduction to Operating Systems}%
			\hfill {Spring 2008} \\
\end{innerlist}
\href{http://www.lums.edu.pk}{\textbf{Lahore University of Management Sciences}} \\
\textit{Teaching Assistant}, CS Dept. \\
\\
Led weekly labs/tutorials and graded programming assignments 

\begin{innerlist}
\item {CS 292: Advanced Programming Techniques}%
			\hfill {Winter 2004-05}
\end{innerlist}

\textit{Lab Instructor}, CS Dept. \\
\\
Designed labs in OPNET simulator 

\begin{innerlist}
\item {CS 471: Computer Networks}%
			\hfill {Spring 2004-05}
\end{innerlist}

\section{Professional Experience}
%
\href{http://www.palmchip.com/}{\textbf{Palmchip Corporation}}, 
Lahore, Pakistan%
			\hfill {May 2005-July 2006}
\begin{innerlist}
\item {Software Engineer, Embedded Systems Group: Optimized bootloader \& filesystem
performances for an in-house System-on-Chip Network-Attached Storage device series.} \\
\end{innerlist}

{\textbf{Syed Murad Ali}, Toronto, Canada}%
			\hfill {Summer 2004}
\begin{innerlist}
\item {Intern, Web Development}
\end{innerlist}

\section{Relevant Coursework, Univ of Pittsburgh (Graduate)}
%
Computer Operating Systems{$^\dagger$}, Computer Architecture{$^\dagger$}, 
Design $\&$ Analysis of Algorithms{$^\dagger$}, Wide Area Networks, 
Computer $\&$ Network Security, Principles of Database Systems, 
Foundations of Artificial Intelligence{$^\dagger$}, Advanced Topics in Operating 
Systems, Secure Software Systems, Advanced Topics in Computer Networks, 
Network Security \\
\\
$\dagger$ \textit{passed preliminary PhD qualifier for the course}

\section{Professional Service}
%
External Reviewer: NSDI 2011, NDSS 2011, CoNEXT 2011 \\
Journal Reviewer: Elsevier Computer Networks Journal \\

\section{Phd Thesis Reviewer}
%
Syed Mohammad Irteza, "Resilient Network Load Balancing for Datacenters", November 2018 \\

\section{Honors}
%
ACM SIGCOMM Travel Grant 2010 \\
Graduate Fellowship Spring 2006 \\
Undergraduate Dean's Honor List 2001-03

\section{Projects}
{\textbf{S{\footnotesize ECURE}E{\footnotesize NVELOPE}}}
        \hfill {Sept 2009-}
\begin{outerlist}
\item[] {A device framework under development in which input keystroke events are securely coupled with actual textual content typed by humans for reliable network payload delivery. This scheme is based on trusted computing principles that places the root of trust on a customized input device running a trusted platform module (TPM) chip and a small attester daemon within it. Each input event generates a cryptographic hash that attests to human activity and the
combined message attestation (derived from such events) gets a third-party verifiable digital signature. These human attestations are then attached to the actual messages which ultimately assist in reducing false positive rates in the recipients' filter modules. Preliminary results to be presented in ApSys 2010 workshop.}
\end{outerlist}
\ \\
{\textbf{B{\footnotesize OTBUSTER}}}%
	\hfill {July 2007-Dec 2008}
\begin{outerlist}
\item[] {DDoS attacks increasingly use normal-looking application-layer requests to waste HTTP server CPU or disk resources. CAPTCHAs attempt to distinguish bots from human clients and are often used to avoid such attacks. However, CAPTCHAs themselves consume resources and frequently are defeated. I developed Bobuster, an extensible ebtables module that pushes client authentication in the kernel while overcoming several limitations in Kill-Bots (NSDI '05). It can easily be deployed as a bridge in front of server farms, modularly accepts a variety of present and future authentication schemes, and can do server-directed client authentication and packet classification. This work was extended as Sentinel and appeared at ICCCN '08 and LCN '10.}
\end{outerlist}

\section{Skills}
C/C++, Java, C$\#$, Python, Javascript, Linux shell scripting, HTML, Unix/GNU Linux

\section{References}
Available on request

\end{document}

%%%%%%%%%%%%%%%%%%%%%%%%%% End CV Document %%%%%%%%%%%%%%%%%%%%%%%%%%%%%
