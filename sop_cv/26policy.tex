\documentstyle[11pt]{article}

\oddsidemargin=.25in
\evensidemargin=.25in
\textwidth=6in
\topmargin=0in
\textheight=9in

\parindent=0in
\pagestyle{empty}

\input{lauracode}

\begin{document}

\centerline {\Large \bf Statement of Purpose}
\vs
%\gotable{0}{1}{rlp{.1in}rl}
%{\bf Instructor:} & Laura Taalman 		 
%   && {\bf Office:}   & 020 Math/Physics \\
%{\bf Telephone:}  & 660-2829 (W), 220-1359 (H) 
%   && {\bf E-mail:}   & taal@math.duke.edu \\
%\stoptable

My first exposure to Computer Science (CS) came during my undergraduate
studies. My interest in the field grew gradually. I discovered that I
was naturally adept at problem-solving using computer
programming. However, I developed a serious interest in the field in
my junior year after I was introduced to Operating Systems and
Computer Network courses. The interest in computer networked systems
grew as I took advanced courses such as Network Security and
Distributed Systems. This led my involvement in a Pakistan
Telecommunication Limited (PTCL) sponsored project on fault-tolerance
using hot-backup schemes [3] for single-threaded application-level
daemons. The work, which also became my undergraduate senior level
project, gave me the first real taste in research which I found
extremely satisfying.\vs

I took a hiatus from academics by joining a local software development
firm, Palmchip Pakistan (Pvt) Ltd (\verb|http://www.palmchip.com|), an
off-shoot of a parent company based in the Silicon Valley. I was
part of an active embedded systems group which was involved in
firmware development on the GNU/Linux based System-On-Chip (SOC)
network-attached storage devices. During that year of job, my
tasks ranged from device driver programming to file systems
performance analysis. Even though my performance was more than
satisfactory at work, I missed the academic life and felt that my
knowledge and interests might be better served elsewhere. The company
did not have an R$\&$D department and my tasks grew monotonous with the
passage of time.
\vs

I decided that a doctoral program was the best option for me (based on
my past experiences) by joining the Computer Science Department of
University of Pittsburgh as a graduate student. Having already worked
in distributed and storage systems, I decided to first explore a wide
range of graduate courses such as Artificial Intelligence, Natural
Language Processing and Database Systems before deciding on a specific
subfield. Along the way, I also cleared all the preliminary PhD
qualifier examinations with the view of achieving a doctoral
degree. The graduate seminar courses honed my technical and analytical
skills to a great degree. 
\vs

I joined the Network Systems Lab as a researcher under the tutelage of
Dr. Jos\'e Carlos Brustoloni, who introduced me to the open problems of
computer and network security. I was a part of a team which was working 
on a web server architecture that can mitigate the effects of Distributed 
Denial-of-Service (DDoS) attacks originating from botnets. I designed, 
implemented and tested an ebtables netfilter extension kernel module 
(a bridging module called Botbuster) which can be installed in front 
of web server farms which audits legitimate HTTP client connections 
through packet scrubbing and deep packet inspection while it blocks 
resource-consuming malicious requests from bots using statistical analysis. 
This work was extended to a project, called Sentinel ([1]). Before Dr.
Brustoloni's departure from the CS department I was working on a project
which has led to a submission of a second paper ([2]) recently. 
\vs

For the last six months I have been working with Dr. KyoungSoo Park in the
Distributed Systems Lab on spam behavior and prevention techniques
exhibited by the CoDeeN Web proxy on the Planetlab network. We have also been 
working on an architecture which reduces false positive rates on spam filters 
using trusted computing techniques. Our work is expected to be finished soon and 
will be submitted to a conference.
%While always having a side interest in storage systems, I have been
%doing some work in distributed object-store file systems under the
%guidance of Dr. Ahmed Amer. This project is, however, still
%work-in-progress as I complete the Botbuster project.
\vs

{\bf A couple of months ago I was informed by the Computer Science department
that my advisor will be leaving next semester.} This prompted me to
analyze my future options. After some discussion with my peers and
faculty members, I came to the conclusion that applying for a doctoral
degree in other universities will be beneficial for me to achieve my
future goals in research and teaching. I have been doing teaching
assistantships every semester since joining the graduate program here
in the University of Pittsburgh. Even in my undergraduate days, I had
been assigned teaching duties during the last few quarters of my
degree program. I see myself quite settled as a teacher and a
researcher in an institution of higher education in the future. My
future goals after attaining the doctoral degree are to gain some
industrial experience by joining a systems research lab before heading
back to Pakistan to join academia there. 
\vs

I was quite thrilled to see the research done by the faculty members in network 
systems \& security areas in the Computer Science department in Carnegie Mellon 
University. Dr. Dave Anderson's and Dr. Adrian Perrig's led work convinced me to 
apply to the doctoral program. These are the kinds of projects I wish to be 
involved in. I believe that the CS department in CMU has a great faculty in 
systems research which would assist me in fulfilling my objectives. I am 
confident that given the opportunity I will be able to contribute positively 
to the department with my dedication and hard work.
\vs
\vs

{\Large \bf References}
\vs

[1] P. Djalaliev, M. Jamshed, N. Farnan and J. Brustoloni. ``Sentinel:
Hardware-Accelerated Mitigation of Bot-Based DDoS Attacks,'' in
\textit{Proceedings of the 17th Intl. Conference on Computer Communications
and Networks (ICCCN'08)}, IEEE, St. Thomas, US Virgin Islands,
Aug. 2008 (AR: 26\%).
\vs

[2] M. Jamshed, J. Brutoloni. ``In-Network Server-Directed Client
Authentication and Packet Classification,'' (submitted).
\vs

[3] Zagorodnov, Dmitrii, Keith Marzullo, Lorenzo Alvisi and Thomas
C. Bressoud. ``Engineering Fault-Tolerant TCP/IP Servers Using
FT-TCP,''in Proceedings of the 2003 Intl. Conference on Dependable
Systems and Networks (DSN'03).
\end{document}
