%%%%%%%%%%%%%%%%%%%%%%%%%%%%%%%%%%%%%%%%%%%%%%%%%%%%%%%%%%%%%%%%%%%%%%%%
%%%%%%%%%%%%%%%%%%%%%% Simple LaTeX CV Template %%%%%%%%%%%%%%%%%%%%%%%%
%%%%%%%%%%%%%%%%%%%%%%%%%%%%%%%%%%%%%%%%%%%%%%%%%%%%%%%%%%%%%%%%%%%%%%%%

%%%%%%%%%%%%%%%%%%%%%%%%%%%%%%%%%%%%%%%%%%%%%%%%%%%%%%%%%%%%%%%%%%%%%%%%
%% NOTE: If you find that it says                                     %%
%%                                                                    %%
%%                           1 of ??                                  %%
%%                                                                    %%
%% at the bottom of your first page, this means that the AUX file     %%
%% was not available when you ran LaTeX on this source. Simply RERUN  %% 
%% LaTeX to get the ``??'' replaced with the number of the last page  %% 
%% of the document. The AUX file will be generated on the first run   %%
%% of LaTeX and used on the second run to fill in all of the          %%
%% references.                                                        %%
%%%%%%%%%%%%%%%%%%%%%%%%%%%%%%%%%%%%%%%%%%%%%%%%%%%%%%%%%%%%%%%%%%%%%%%%

%%%%%%%%%%%%%%%%%%%%%%%%%%%% Document Setup %%%%%%%%%%%%%%%%%%%%%%%%%%%%

% Don't like 10pt? Try 11pt or 12pt
\documentclass[10pt]{article}

% This is a helpful package that puts math inside length specifications
\usepackage{calc}

% Layout: Puts the section titles on left side of page
\reversemarginpar

%
%         PAPER SIZE, PAGE NUMBER, AND DOCUMENT LAYOUT NOTES:
%
% The next \usepackage line changes the layout for CV style section
% headings as marginal notes. It also sets up the paper size as either
% letter or A4. By default, letter was used. If A4 paper is desired,
% comment out the letterpaper lines and uncomment the a4paper lines.
%
% As you can see, the margin widths and section title widths can be
% easily adjusted.
%
% ALSO: Notice that the includefoot option can be commented OUT in order
% to put the PAGE NUMBER *IN* the bottom margin. This will make the
% effective text area larger.
%
% IF YOU WISH TO REMOVE THE ``of LASTPAGE'' next to each page number,
% see the note about the +LP and -LP lines below. Comment out the +LP
% and uncomment the -LP.
%
% IF YOU WISH TO REMOVE PAGE NUMBERS, be sure that the includefoot line
% is uncommented and ALSO uncomment the \pagestyle{empty} a few lines
% below.
%

%% Use these lines for letter-sized paper
\usepackage[paper=letterpaper,
            %includefoot, % Uncomment to put page number above margin
            marginparwidth=1.2in,     % Length of section titles
            marginparsep=.05in,       % Space between titles and text
            margin=0.25in,            % 0.25 inch margins
            includemp]{geometry}

%% Use these lines for A4-sized paper
%\usepackage[paper=a4paper,
%            %includefoot, % Uncomment to put page number above margin
%            marginparwidth=30.5mm,    % Length of section titles
%            marginparsep=1.5mm,       % Space between titles and text
%            margin=25mm,              % 25mm margins
%            includemp]{geometry}

%% More layout: Get rid of indenting throughout entire document
\setlength{\parindent}{0in}

%% This gives us fun enumeration environments. compactitem will be nice.
\usepackage{paralist}

%% Reference the last page in the page number
%
% NOTE: comment the +LP line and uncomment the -LP line to have page
%       numbers without the ``of ##'' last page reference)
%
% NOTE: uncomment the \pagestyle{empty} line to get rid of all page
%       numbers (make sure includefoot is commented out above)
%
\usepackage{fancyhdr,lastpage}
\pagestyle{fancy}
%\pagestyle{empty}      % Uncomment this to get rid of page numbers
\fancyhf{}\renewcommand{\headrulewidth}{0pt}
\fancyfootoffset{\marginparsep+\marginparwidth}
\newlength{\footpageshift}
\setlength{\footpageshift}
          {0.5\textwidth+0.5\marginparsep+0.5\marginparwidth-2in}
\lfoot{\hspace{\footpageshift}%
       \parbox{4in}{\, \hfill %
%                    \arabic{page} of \protect\pageref*{LastPage} % +LP
                    \arabic{page}                               % -LP
                    \hfill \,}}

% Finally, give us PDF bookmarks
\usepackage{color,hyperref}
\definecolor{mulberry}{rgb}{0.77,0.29,0.55}
\hypersetup{colorlinks,breaklinks,
            linkcolor=mulberry,urlcolor=mulberry,
            anchorcolor=mulberry,citecolor=mulberry}
%\definecolor{darkblue}{rgb}{0.0,0.0,0.3}
%\hypersetup{colorlinks,breaklinks,
%            linkcolor=darkblue,urlcolor=darkblue,
%            anchorcolor=darkblue,citecolor=darkblue}

%%%%%%%%%%%%%%%%%%%%%%%% End Document Setup %%%%%%%%%%%%%%%%%%%%%%%%%%%%


%%%%%%%%%%%%%%%%%%%%%%%%%%% Helper Commands %%%%%%%%%%%%%%%%%%%%%%%%%%%%

% The title (name) with a horizontal rule under it
%
% Usage: \makeheading{name}
%
% Place at top of document. It should be the first thing.
\newcommand{\makeheading}[1]%
        {\hspace*{-\marginparsep minus \marginparwidth}%
         \begin{minipage}[t]{\textwidth+\marginparwidth+\marginparsep}%
                {\large \bfseries #1}\\[-0.15\baselineskip]%
                 \rule{\columnwidth}{1pt}%
         \end{minipage}}

% The section headings
%
% Usage: \section{section name}
%
% Follow this section IMMEDIATELY with the first line of the section
% text. Do not put whitespace in between. That is, do this:
%
%       \section{My Information}
%       Here is my information.
%
% and NOT this:
%
%       \section{My Information}
%
%       Here is my information.
%
% Otherwise the top of the section header will not line up with the top
% of the section. Of course, using a single comment character (%) on
% empty lines allows for the function of the first example with the
% readability of the second example.
\renewcommand{\section}[2]%
        {\pagebreak[2]\vspace{1.2\baselineskip}%
         \phantomsection\addcontentsline{toc}{section}{#1}%
         \hspace{0in}%
         \marginpar{
         \raggedright \scshape #1}#2}

% An itemize-style list with lots of space between items
\newenvironment{outerlist}[1][\enskip\textbullet]%
        {\begin{itemize}[#1]}{\end{itemize}%
         \vspace{-.6\baselineskip}}

% An environment IDENTICAL to outerlist that has better pre-list spacing
% when used as the first thing in a \section 
\newenvironment{lonelist}[1][\enskip\textbullet]%
        {\vspace{-\baselineskip}\begin{list}{#1}{%
        \setlength{\partopsep}{0pt}%
        \setlength{\topsep}{0pt}}}
        {\end{list}\vspace{-.6\baselineskip}}

% An itemize-style list with little space between items
\newenvironment{innerlist}[1][\enskip\textbullet]%
        {\begin{compactitem}[#1]}{\end{compactitem}}

% To add some paragraph space between lines.
% This also tells LaTeX to preferably break a page on one of these gaps
% if there is a needed pagebreak nearby.
\newcommand{\blankline}{\quad\pagebreak[2]}

%%%%%%%%%%%%%%%%%%%%%%%% End Helper Commands %%%%%%%%%%%%%%%%%%%%%%%%%%%

%%%%%%%%%%%%%%%%%%%%%%%%% Begin CV Document %%%%%%%%%%%%%%%%%%%%%%%%%%%%

\begin{document}
\newlength{\rcollength}\setlength{\rcollength}{2.60in}%
\begin{tabular}[t]{@{}p{\textwidth-\rcollength}p{\rcollength}}
\makeheading{M{\footnotesize UHAMMAD} A{\footnotesize SIM} J{\footnotesize AMSHED}} & \hspace{0.75in}{\it Last Updated: 09/17/2019}
\end{tabular}

\section{Contact Information}
%
% NOTE: Mind where the & separators and \\ breaks are in the following
%       table.
%
% ALSO: \rcollength is the width of the right column of the table 
%       (adjust it to your liking; default is 1.85in).
%
%\newlength{\rcollength}\setlength{\rcollength}{2.60in}%
%
\begin{tabular}[t]{@{}p{\textwidth-\rcollength}p{\rcollength}}
Intel Jones Farm 2 (JF2) Building, 
& \textit{E-mail:}
\href{mailto:asim.jamshed@gmail.com}{\small{\texttt{asim.jamshed@gmail.com}}} \\
2111 NE 25th Ave, Hillsboro, OR 97124    & \textit{WWW:}
\href{https://ajamshed.github.io}{\small{\texttt{ajamshed.github.io}}}\\
\end{tabular}

\section{Interests}
%
Networked systems design \& implementation, distributed systems, network security and operating systems.

\section{Education}
%
{\textbf{Korea Advanced Institute of Science \& Technology (KAIST)}}, Republic of Korea
\begin{innerlist}
\item {PhD, {Electrical Engineering} (Spring 2017). Advisor -- Prof. KyoungSoo Park}
\end{innerlist}

{\textbf{University of Pittsburgh}}, Pittsburgh, Pennsylvania, USA
\begin{innerlist}
\item {MS, {Computer Science} (Apr 2010). Advisors -- Prof. KyoungSoo Park \& Prof. Daniel Moss\'e}
\end{innerlist}

{\textbf{Lahore University of Management Sciences}}, Pakistan
\begin{innerlist}
\item {BSc (Hons), {Computer Science}, (May 2005).}
%\item {Minor in Mathematics}
\end{innerlist}

\section{Employment Experience (Selected)}
%
{\textbf{Intel Labs, Intel Jones Farm 2 (JF2)}, Hillsboro, OR}
\begin{innerlist}
\item {Research Scientist, Telco Systems (May 2017-onwards). Reporting to Christian Maciocco (Principal Engineer)}
\end{innerlist}

{\textbf{International Computer Science Institute (ICSI)}, Berkeley, CA} 
\begin{innerlist}
\item {Research Intern (May 2014-Aug 2014, Oct 2015-Dec 2015). Mentor -- Dr. Robin Sommer}
\item {Developed Packet Bricks. See [3] in Projects section.}
\end{innerlist}

{\textbf{Palmchip Corporation}, Lahore, Pakistan}
\begin{innerlist}
\item {Software Engineer (May 2005-July 2006). Reporting to Ahrar Naqvi (VP Engineering)}
\item {Optimized bootloader \& filesystem performances for an system-on-chip network-attached storage device series.}
\end{innerlist}

\section{Projects/ Software (Selected)}
{\textbf{1. O{\footnotesize MEC} P{\footnotesize ROJECT}}} (\href{https://github.com/omec-project/ngic-rtc}{\texttt{https://github.com/omec-project/ngic-rtc}})
\begin{innerlist}
\item {Control User Plane Separated (CUPS) {\texttt TS23501} based EPC Service \& Packet Gateways (SGW, PGW)}
\item {\textit{URL:} \href{https://www.opennetworking.org/omec/}{\texttt{https://www.opennetworking.org/omec/}}}
\end{innerlist}

{\textbf{2. m{OS} S{\footnotesize TACK}}} (\href{https://github.com/ndsl-kaist/mOS-networking-stack}{\texttt{https://github.com/ndsl-kaist/mOS-networking-stack}})
\begin{innerlist}
\item {A Specialized Network Programming Library for Stateful Middelboxes.}
\item {\textit{Pub:} {\bf NSDI 2017}, \textit{URL:} \href{http://mos.kaist.edu/}{\texttt{http://mos.kaist.edu/}}}
\item {{\bf NSDI Best Paper Award 2017.}}  
\end{innerlist} 

{\textbf{3. P{\footnotesize ACKET} B{\footnotesize RICKS}}} (\href{https://github.com/bro/packet-bricks}{\texttt{https://github.com/bro/packet-bricks}})
\begin{innerlist}
\item {A netmap-based packet layer for distributing and filtering traffic.}
\end{innerlist} 

{\textbf{4. m{TCP}}} (\href{https://github.com/eunyoung14/mtcp/}{\texttt{https://github.com/eunyoung14/mtcp/}})
\begin{innerlist}
\item {A Highly Scalable User-level TCP Stack for Multicore Systems.}
\item {{\bf NSDI Community Award 2014}, Runner-up {\bf Samsung HumanTech Paper Award 2014}.}
\item {\textit{Pub:} {\bf NSDI 2014}, \textit{URL:} \href{http://shader.kaist.edu/mtcp/}{\texttt{http://shader.kaist.edu/mtcp/}}}
\end{innerlist}

{\textbf{5. K{\footnotesize ARGUS}}}
\begin{innerlist}
\item {A Highly-scalable Software-based Network Intrusion Detection System.}
\item {{\bf ``10 Achievements of 2012 that put KAIST on the Spotlight.''}}
\item {\textit{Pub:} {\bf CCS 2012}, \textit{URL:} \href{http://shader.kaist.edu/kargus/}{\texttt{http://shader.kaist.edu/kargus/}}}
\end{innerlist}

\section{Publications (Selected)}
[1] {``AccelTCP: Accelerating Network Applications with Stateful TCP Offloading.''}
   NSDI '20

[2] {``Reducing Tail Latency via Safe and Simple Duplication.''}
   CoNEXT '19

[3] {``mOS: A Reusable Networking Stack for Flow Monitoring Middleboxes.'' }
   NSDI '17 - {\bf Best Paper Award}

[4] {``APUNet: Revitalizing GPU as Packet Processing Accelerator.'' }
   NSDI '17

[5] {``DFC: Accelerating String Pattern Matching for Network Applications.'' }
   NSDI '16

[6] { ``Haetae: Scaling the Performance of Network Intrusion Detection with Many-core Processors.'' }
   RAID '15
%
%[5] { ``A Case for a Stateful Middlebox Networking Stack." }
%    SIGCOMM CCR 2015

%[5.5] { ``Scaling the Performance of Network Intrusion Detection with Many-core Processors." }
%    ANCS 2015 (Poster)
%
[7] { ``mTCP: a Highly Scalable User-level TCP Stack for Multicore Systems.'' }
    NSDI '14 - {\bf Community Award}

[8] { ``Kargus: a Highly-scalable Software-based Intrusion Detection System.'' }
    CCS '12

[9] { ``Suppressing Bot Traffic with Accurate Human Attestations.'' }
    ApSys '10

[10] {``Sentinel: Hardware-Accelerated Mitigation of Bot-Based DDoS Attacks.''}
    ICCCN '08
%
%[10] {``In-Network Server-Directed Client Authentication
%		and Packet Classification."}
%     LCN '10

%\section{Relevant Coursework}
%
%Computer Operating Systems, Computer Architecture, 
%Design $\&$ Analysis of Algorithms, Wide Area Networks, 
%Computer $\&$ Network Security, Principles of Database Systems, 
%Foundations of Artificial Intelligence, Advanced Topics in Operating 
%Systems, Secure Software Systems, Advanced Topics in Computer Networks, 
%Network Security, Think Like an Adversary, Performance Analysis of Communication Networks,
%Parallel and Distributed Computation in Communication Network, Software-defined Networked Computing
%
\section{Awards}
%
NSDI Best Paper Award 2017 for mOS \\
2$^{nd}$ Runner-up Samsung Humantech Paper Award 2016 for DFC \\
NSDI Community Award 2014, \& Runner-up Samsung Humantech Paper Award 2014 for mTCP \\
``10 Achievements of 2012 that put KAIST on the Spotlight'' for Kargus \\
%ACM SIGCOMM Travel Grant 2010 \\
Graduate Fellowship Spring 2006 \\
Undergraduate Dean's Honor List 2001-03

\section{Skills}
C/C++, C$\#$, Java, Python, CUDA, Lua, Javascript, HTML/XML, Linux, x86 Assembly, TILE-Gx, Intel DPDK \\
%
\end{document}

%%%%%%%%%%%%%%%%%%%%%%%%%% End CV Document %%%%%%%%%%%%%%%%%%%%%%%%%%%%%
